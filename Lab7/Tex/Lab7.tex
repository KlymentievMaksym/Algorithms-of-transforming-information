\documentclass{article}
\usepackage[utf8]{inputenc}
\usepackage[fontsize = 12pt]{fontsize} 
\usepackage{microtype}
\usepackage[english, ukrainian]{babel}
\usepackage[
    top = 40pt,
    bottom = 20pt,
    right = 50pt,
    left = 50pt
]{geometry}
\usepackage{biblatex}
\usepackage{csquotes}
\usepackage{mathtools}

\begin{document}
    \begin{titlepage}
        \begin{center}
        $\newline$
        \vspace{3.3cm}
        
        {\LARGE\textbf{Лабораторна робота №7\\"Використання префiксного дерева для розв’язання задачi множинної вiдповiдностi шаблонам"}}
        \vspace{10cm}
        \begin{flushright}
            \textbf{Роботу виконав:}\\Климентьєв Максим \\3-го курсу\\групи ФІ-21
        \end{flushright}
        \end{center}
    \end{titlepage}
    \newpage

    \pagenumbering{gobble}
    \tableofcontents
    \cleardoublepage
    \pagenumbering{arabic}
    \setcounter{page}{3}
    \newpage

    \section{Особливості реалізації}
    Нижче наведені особливості реалізації:
    \begin{itemize}
        \item \textbf{Аби було} --- вивід, які патерни присутні в префіксному дереві (в не дуже зрозумілому мені порядку).
        \item \textbf{Додавання змінної text} --- аби вивести індекси де починається якийсь префікс.
        \item \textbf{Додавання ключів key та key\_add до TrieNode} --- аби вивести індекси з якого індексу починається та куди потрапляє певна буква.
        \item \textbf{Вивід} --- відбувається з кінцевої букви у словнику дітей і йде вивід аж до самої першої дитини. Можна інвертувати, якщо робити stack.pop(0) замість stack.pop()
    \end{itemize}    

    \section{Приклади}
    \begin{enumerate}
        \item $ \text{Вхiднi данi} (input\_1.1.dat):\\
            1\\
            CGT\\
            \text{Вихiднi данi} (output\_1.1.dat):\\
            0\rightarrow1:C\\
            1\rightarrow2:G\\
            2\rightarrow3:T $
        \item $ \text{Вхiднi данi} (input\_1.2.dat):\\
            3\\
            CG\\
            CC\\
            CT\\
            \text{Вихiднi данi} (output\_1.2.dat):\\
            0\rightarrow1:C\\
            1\rightarrow2:G\\
            1\rightarrow3:C\\
            1\rightarrow4:T $
        \item $ \text{Вхiднi данi} (input\_2.1.dat):\\
            CCCC\\
            1\\
            CC\\
            \text{Вихiднi данi} (output\_2.1.dat):\\
            0\ 1\ 2$
        \item $ \text{Вхiднi данi} (input\_2.2.dat):\\
            CCCC\\
            2\\
            CG\\
            CT\\
            \text{Вихiднi данi} (output\_2.2.dat):\\
            \\
            \text{Що є порожнiм файлом.}$
        \item $ \text{Вхiднi данi} (input\_2.3.dat):\\
            ATTCCGATA\\
            2\\
            AT\\
            C\\
            \\
            \text{Вихiднi данi} (output\_2.3.dat):\\
            0\ 3\ 4\ 6$
    \end{enumerate}
\end{document}