\documentclass{article}
\usepackage[utf8]{inputenc}
\usepackage[fontsize = 12pt]{fontsize} 
\usepackage{microtype}
\usepackage[english, ukrainian]{babel}
\usepackage[
    top = 40pt,
    bottom = 20pt,
    right = 50pt,
    left = 50pt
]{geometry}
\usepackage{biblatex}
\usepackage{csquotes}
\usepackage{mathtools}

\begin{document}
    \begin{titlepage}
        \begin{center}
        $\newline$
        \vspace{3.3cm}
        
        {\LARGE\textbf{Лабораторна робота №8\\"Застосування перетворення Берроуза-Вiлера для розв’язання задачi вiдповiдностi шаблону"}}
        \vspace{10cm}
        \begin{flushright}
            \textbf{Роботу виконав:}\\Климентьєв Максим \\3-го курсу\\групи ФІ-21
        \end{flushright}
        \end{center}
    \end{titlepage}
    \newpage

    \pagenumbering{gobble}
    \tableofcontents
    \cleardoublepage
    \pagenumbering{arabic}
    \setcounter{page}{3}
    \newpage

    \section{Особливості реалізації}
    Нижче наведені особливості реалізації:
    \begin{itemize}
        \item \textbf{Використання Node} --- для пов'язання між собою літер.
        \item \textbf{Усе в одному} --- в залежності від параметрів робить необхідну дію.
        \item \textbf{Не вистачає пам'яті для розміру вище (1.8e4 + 1)} --- Скоріше за все через погану реалізацію прямого метода.
        \item \textbf{Повільна реверсія} --- є два моменти, перше це генерація списку з нодів, друге - прохід по словам, щоб утворити обернене слово.
    \end{itemize}
\end{document}